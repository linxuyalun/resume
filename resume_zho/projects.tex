%-------------------------------------------------------------------------------
%	SECTION TITLE
%-------------------------------------------------------------------------------
\cvsection{项目}


%-------------------------------------------------------------------------------
%	CONTENT
% STAR: Situation -> Task -> Action -> Result
%-------------------------------------------------------------------------------
\begin{cventries}

%---------------------------------------------------------
  \cventry
    {项目负责人} % Job title
    % By Siyuan
    % 我认为可以在最后添加一条这个项目具体的价值,比如估值多少啦,承接了多少个重要项目啦,给一些具体的数字,
    % 同时除了项目负责人这一点,我觉得可以多写一下你在这个项目中的重要度,不然可能会被低估 both project and you
    {Casecloud} % Project Name
    {横向项目} % Location
    {2019年5月 至 今} % Date(s)
    {
      \begin{cvitems} % Description(s) of tasks/responsibilities
        \setlength\itemsep{1mm}
        % TODO: 扩写一下 CC 的使用情况
        \item {Casecloud 是一个基于云原生的,服务数百名律师与数千名债权人的破产管理系统。}
        \item {负责技术选型,原型搭建和项目管理:考量项目与团队的特点,分割业务边界,采用微服务架构;采用GitHub和Jira作为代码管理和项目管理,把控项目进度和风险;带领团队在项目初期完成多次项目演示与测试,上线以来发布接口 200 余个,控制每个接口的响应时间,保证服务质量。}
        \item {负责效能管理:运维方面,约定了项目版本号,搭建和管理了一套测试集群,定义了一些规约更好的区分了开发环境和测试环境,帮助开发人员更好的开发测试应用业务的同时不会因为人为因素造成混乱;前端方面,统一使用函数式编程风格开发,约定代码规范,推进团队使用GitHub Flow的模式进行代码开发,并负责代码审查,保证项目的代码质量。}
        \item {负责搭建运维体系和自动化流程:使用Kubernetes集群进行容器编排,管理集群中近30个微服务,并根据服务情况动态扩缩容;使用Istio接管南北流量和东西流量,高效规范的管理不同微服务与不同版本之间的流量规则,并在Istio层面配置服务熔断和金丝雀发布,保证项目的服务质量;在团队内搭建和维护 CI/CD pipeline,缩短反馈闭环周期;搭建监控与日志服务,提供集群和应用较好的观测性;使用K8s部署管理Mongo集群,提供灵活的数据管理。}
        \item {负责前端的设计与实现:设计与实现基于React,Redux和TypeScript的前端;设计代理中心控制前端的网络请求,使用中间件模型和拦截器管理权限token和请求响应的统一处理,保证了前端清晰的架构并做到易于扩展,降低开发人员随着项目复杂程序增加的开发难度。}
        \item {负责后端IAM的设计与实现:一个IAM服务,同时配合Istio的流量管理,提供一个中心式的鉴权管理,做到与其他服务解耦,鉴权服务作为项目入口的安全网关保证了项目业务上的安全可靠。}
      \end{cvitems}
    }

%---------------------------------------------------------

\end{cventries}
