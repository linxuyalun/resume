%-------------------------------------------------------------------------------
%	SECTION TITLE
%-------------------------------------------------------------------------------
\cvsection{项目经历}


%-------------------------------------------------------------------------------
%	CONTENT
%-------------------------------------------------------------------------------
\begin{cventries}

%---------------------------------------------------------
  \cventry
    {项目负责人} % Job title
    {Casecloud} % Project Name
    {横向项目} % Location
    {2019年5月 至 今} % Date(s)
    {
    % TODO: language used in experiences
      \begin{cvitems} % Description(s) of tasks/responsibilities
        % TODO: 可以加上一些效果啊啥的。
        \setlength\itemsep{1mm}
        \item {Casecloud 是一个基于云原生的,服务数百名律师与数千名债权人的破产管理系统。}
        % 具化微服务设计的考量
        \item {负责技术选型,原型搭建和项目管理:考量项目与团队的特点,分割业务边界,采用微服务架构;采用GitHub和Jira作为代码管理和项目管理,把控项目进度和风险;带领团队在项目初期完成多次项目演示与测试,上线以来发布接口 200 余个,控制每个接口的响应时间,保证服务质量。}
        \item {负责从0到1的项目运维:使用Kubernetes集群进行容器编排,管理集群中近30个微服务,并根据服务情况动态扩缩容;搭建和管理了一套测试集群,定义了一些规约更好的区分了开发环境和测试环境,帮助开发人员更好的开发测试应用业务的同时不会因为人为因素造成混乱;使用Istio接管集群与外部服务的南北流量和服务间的东西流量,高效规范的管理不同微服务与不同版本之间的流量规则,并在Istio层面配置了服务熔断和金丝雀发布,保证项目的服务质量;在团队内搭建和维护 CI/CD pipeline,缩短反馈闭环周期;搭建Prometheus负责集群的监控,搭建ELK负责日志收集与查看,提供集群和应用较好的观测性;使用Kubernetes部署管理Mongo集群,提供灵活的管理项目数据,并在开发期间实现了一次数据的快速迁移。}
        \item {负责前端的设计与实现:设计与实现基于React,Redux和TypeScript的前端开发;设计代理中心控制前端的网络请求,使用中间件模型和拦截器管理权限token和请求响应的统一处理,保证了前端清晰的架构并做到易于扩展,降低开发人员随着项目复杂程序增加的开发难度;统一使用函数式编程风格开发,约定代码规范,推进团队使用GitHub Flow的模式进行代码开发,并负责代码审查,保证项目的代码质量。}
        \item {负责后端部分微服务的设计与实现:一个中心式的鉴权服务,提供一种配置规范方便其他开发者表达接口权限,利用Envoy流量拦截,将发往各个接口的流量提前拦截下来,根据返回结果判定是否继续发送给相应服务或者直接返回,从而做到与其他服务解耦,鉴权服务作为项目入口的安全网关保证了项目的安全可靠;一个Excel表格下载服务,通过和其他微服务交互获取数据,并返回Excel表格(这个服务是否有必要在简历里写?)。}
      \end{cvitems}
    }

%---------------------------------------------------------

%---------------------------------------------------------
  \cventry
  {核心后端开发人员} % Job title
  {人流监控系统} % Project Name
  {横向项目} % Location
  {2020年11月 至 2021年2月} % Date(s)
  {
  % TODO: language used in experiences
    \begin{cvitems} % Description(s) of tasks/responsibilities
      % TODO: 可以加上一些效果啊啥的。
      \setlength\itemsep{1mm}
      \item {负责后端系统的设计与实现和优化:负责系统监控和业务的接口的设计与实现,监控包含边缘设备本身的监控和业务数据的监控,前者利用 shell 脚本采集系统多种维度的数据,并提供可配置的阈值设置,后者接收其他服务的报警数据并进行持久化和统计;实现图像数据传输的传输优化,减少了数据的额外封装,将每帧4K图像传输至图像处理服务的总耗时从300毫秒优化至20毫秒。}
      \item {负责优化人物监测服务:由于服务本身算法的性能上限无法应对多个视频流的图像数据,利用额外的数据结构和元数据保证算法的有序处理,不会发生饥饿;重新设计与优化判定人物跌倒和徘徊的处理逻辑,使其能在不依赖视频流数据连续性的情况下进行逻辑判定,添加了配置选项,提高了服务的易扩展性。}
    \end{cvitems}
  }

%---------------------------------------------------------

\end{cventries}
