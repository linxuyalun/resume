%-------------------------------------------------------------------------------
%	SECTION TITLE
%-------------------------------------------------------------------------------
\cvsection{实习}


%-------------------------------------------------------------------------------
%	CONTENT
%-------------------------------------------------------------------------------
\begin{cventries}

%---------------------------------------------------------
  \cventry
    {PaaS 镜像仓库后端实习生} % Job title搭建
    {ByteDance} % Organization
    {上海} % Location
    {2021年6月 至 今} % Date(s)
    {
      \begin{cvitems} % Description(s) of tasks/responsibilities
        \setlength\itemsep{1mm}
        \item {全地域镜像仓库实例自动化管理:为支持全地域多镜像仓库实例的创建与销毁,参与设计了镜像仓库实例自动化管理整体架构;数据面设计并实现了镜像仓库 Operator,实现了单地域下镜像实例的创建,销毁和服务状态的维护;控制面设计并实现了镜像仓库 Manager,下发镜像实例的创建销毁指令并监控实例的健康状态。}
        \item {项目优化与文档产出:积极推动项目整体代码重构与文档产出。分析目前项目存在的问题以及可优化项,逐项优化,提升项目整体的可扩展性和简明性;实习期间产出多篇项目技术文档,以方案为导向,提升流程规范,降低沟通成本;补全大量测试用例,提升项目整体的可靠性。}
        \item {镜像仓库业务支持:实习期间支持业务项目整体的推进,包括支持 Helm Chart 的 OCI 制品,部分接口的性能优化,业务代码 bug 修复与需求变更调整。}
      \end{cvitems}
    }

%---------------------------------------------------------

% %---------------------------------------------------------
%   \cventry
%     {Outreach COO 实习} % Job title搭建
%     {MSRA} % Organization
%     {北京} % Location
%     {2018年6月 至 2018年9月} % Date(s)
%     {
%     % TODO: language used in experiences
%       \begin{cvitems} % Description(s) of tasks/responsibilities
%         % TODO: 可以加上一些效果啊啥的。
%         \setlength\itemsep{1mm}
%         \item {2018微软学生夏令营的策划与筹办:微软夏令营是面向全国34所高校的活动,每年举办一次。负责活动前期所有的策划工作与活动期间的执行工作。工作包括设计黄金点挑战赛,筹划Hack for Earth,收到营员和研究员的一致好评;筹划Hack for Earth,让营员们体验紧张刺激的Hackathon比赛;邀请副院长,微软学者,研究员进行演讲,介绍前沿技术;策划举办MSRA 20周年庆。}
%         \item {获得优秀实习生“明日之星”称号。}
%       \end{cvitems}
%     }

% %---------------------------------------------------------

\end{cventries}
